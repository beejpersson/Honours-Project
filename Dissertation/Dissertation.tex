	%You can delete all the comments after you have finished your document
%this sets up the defaults for the documents, 12pt font and A4 size. The article type sets this up as such as opposed to letter or memo.

%for the finer points LaTeX see https://en.wikibooks.org/wiki/LaTeX or http://tex.stackexchange.com/

\documentclass[12pt,a4paper]{article}
\usepackage{titlesec} %these are how we import packages, one helps set up footers and title layout
\usepackage{fancyhdr}

% !TEX TS-program = pdflatex
% !TEX encoding = UTF-8 Unicode
\usepackage[utf8]{inputenc} % set input encoding (not needed with XeLaTeX)

\usepackage{graphicx} % support the \includegraphics command and options

% \usepackage[parfill]{parskip} % Activate to begin paragraphs with an empty line rather than an indent

%%% PACKAGES
\usepackage{booktabs} % for much better looking tables
\usepackage{array} % for better arrays (eg matrices) in maths
\usepackage{paralist} % very flexible & customisable lists (eg. enumerate/itemize, etc.)
\usepackage{verbatim} % adds environment for commenting out blocks of text & for better verbatim
\usepackage{hyperref} % adds \url command for hyperlinks in text
\usepackage{subfig} % make it possible to include more than one captioned figure/table in a single float
\usepackage[toc,page]{appendix}
% These packages are all incorporated in the memoir class to one degree or another...

%header and footer settings
\pagestyle{fancyplain}
\fancyhf{}
\renewcommand{\headrulewidth}{0.5pt}
\renewcommand{\footrulewidth}{0.5pt}
\setlength{\headheight}{15pt}
\fancyhead[L]{Beej Persson - 40183743}
\fancyhead[R]{SOC10101 Honours Project}
\fancyfoot[L]{}
\fancyfoot[C]{\thepage}

%set better section layout
\makeatletter
\renewcommand\subsection{\@startsection {subsection}{1}{2mm} % name, level, indent
                               {3pt plus 2pt minus 1pt} % before skip
                               {3pt plus 0pt} % after skip
                               {\normalfont\bfseries}}
\makeatother
\makeatletter
\renewcommand\section{\@startsection {section}{1}{0mm} % name, level, indent
                               {4pt plus 2pt minus 1pt} % before skip
                               {4pt plus 0pt} % after skip
                               {\bfseries}}
\makeatother


%this starts the document
\begin{document}

%you can import other documents into your main one, these layout the Title and Declarations on its own page.
%you might need to change these to \ if your on Microsoft Windows.
\newcommand{\HRule}{\rule{\linewidth}{0.5mm}}

\begin{titlepage}
	\begin{center}

	\HRule \\[0.4cm]
    	{\Large \bfseries Real-World Object Capture in a \\ Mixed Reality Environment\par}
	\vspace{0.2cm}
	\HRule \\[1.5cm]

	
    	\vspace{3cm}
	\begin{minipage}{0.4\textwidth}
	\begin{center} \large
        \emph{}\\
        	Beej Persson - 40183743
				
   	 \end{center}
    	\end{minipage}
	
	\vspace{2cm}
    	\begin{minipage}{1\textwidth}
    	\begin{center} \large
        
		Submitted in partial fulfilment of \\
		the requirements of Edinburgh Napier University \\
		for the Degree of \\
        	BSc (Hons) Games Development
    	\end{center}
    	\end{minipage}

    	\vfill

    	% Bottom of the page
	\begin{minipage}{1\textwidth}
    	\begin{center} \large
		School of Computing
    	\end{center}
    	\end{minipage}
	
	\vspace{1cm}
    	{\large \today}


	\end{center}
\end{titlepage}
%{\large Submitted in partial fulfilment of the requirements of Edinburgh Napier University for the Degree of }

\section*{Authorship Declaration}
\vspace{0.5cm}
\begin{flushleft}
I, Beej Persson, confirm that this dissertation and the work presented in it are my own achievement.\newline

Where I have consulted the published work of others this is always clearly attributed;\newline

Where I have quoted from the work of others the source is always given. With the exception of such quotations this dissertation is entirely my own work;\newline

I have acknowledged all main sources of help; \newline

If my research follows on from previous work or is part of a larger collaborative research project I have made clear exactly what was done by others and what I have contributed myself;\newline

I have read and understand the penalties associated with Academic Misconduct.\newline

I also confirm that I have obtained informed consent from all people I have involved in the work in this dissertation following the School's ethical guidelines.\newline
\end{flushleft}

\begin{flushleft} \large
\emph{Signed:} \\
\end{flushleft}

\vspace{.5cm}

\begin{flushleft} \large
\emph{Date:} \\
\end{flushleft}

\vspace{.5cm}

\begin{flushleft} \large
\emph{Matriculation no: }  \\
\end{flushleft}
\pagebreak

\section*{Data Protection Declaration}
\vspace{0.5cm}
\begin{flushleft}
Under the 1998 Data Protection Act, The University cannot disclose your grade to an unauthorised person. However, other students benefit from studying dissertations that have their grades attached. \newline

\vspace{0.5cm}

Please sign your name below one of the options below to state your preference.\newline
\vspace{0.5cm}

The University may make this dissertation, with indicative grade, available to others.\newline
\vspace{3cm}


The University may make this dissertation available to others, but the grade may not be disclosed.\newline
\vspace{3cm}


The University may not make this dissertation available to others.\newline
\end{flushleft}


\pagebreak

%LaTeX let you define the abstract separately so it wont get sucked into the main document.
\begin{abstract}
% fill the abstract in here
\end{abstract}
\pagebreak

\tableofcontents % is generated for you
\newpage

\listoftables
%generated in same way as figures
\newpage

\listoffigures
%you may have captions such as equations, listings etc they should all appear as required
%these are done for you as long as you use \begin{figure}[placement settings] .. bla bla ... \end{figure}
\newpage

\section*{Acknowledgements}
Insert acknowledgements here
\subsection*{}
	I would like to thank my widely-desired work ethic, talk-of-the-town dedication and my optimistic opinions of my abilities.
\newpage


\section{Introduction}
\subsection{Background}
This project has three main background components to discuss. These are mixed reality, the Microsoft HoloLens and Vuforia.
\subsubsection{Mixed Reality}
Mixed reality (MR) is a variation of virtual environments, or virtual reality. A virtual environment is one in which a user is immersed inside a synthetic environment. Whilst in this environment the user cannot see the real world around them. \cite{lamport94}
\subsection{Aims and Objectives}

\subsubsection{Overview Of Project Content and Milestones}

This is a sub sub section with a list of bullet points.
\begin{itemize}\itemsep0pt
	\item A working X, that will be used for this investigation.
	\item Investigation of current tools and their potential use during an investigation of X .
	\item Programming of X with related frameworks Y and Z.
	\item That is all.
\end{itemize}


\subsection{Scope and Limitations}

\subsection{Structure of this Dissertation}

\section{Methodology}
And so on for each of the chapters.  The template automatically starts new chapters on a new page.  The associated guidelines tell you what the available styles do and also how to structure a report.
There is a section break on this page that you should be careful NOT to delete otherwise the references and appendices will be numbered continuously with the rest of the document.

\section{Evaluation}

\section{Conclusion}
\subsection{Future Work}

% another example section
\section{Additional Information / Knowledge Required}
Experience with Linux and managing Virtual machines, networking.
So on and so forth...

\newpage
\begin{thebibliography}{9}

\bibitem{lamport94}
  Leslie Lamport,
  \emph{\LaTeX: A Document Preparation System}.
  Addison Wesley, Massachusetts,
  2nd Edition,
  1994.

\end{thebibliography}
%example of References. See https://en.wikibooks.org/wiki/LaTeX/Bibliography_Management
%might be good to use a separate document for these so your main work is not one really long text file. 

%you can crate this on a extra tex document just like the title or any other part of the document.
\newpage
\begin{appendices}
\section{Initial Project Overview}
 \ \\ \textbf{SOC10101 Honours Project (40 Credits)} \\ \\
\textbf{Title of Project:} \\
Real world object-capture in a mixed-reality environment. \\ \\
\textbf{Overview of Project Content and Milestones:} \\
This project will look to evaluate the effectiveness of capturing real world objects using Vuforia on the Microsoft HoloLens. Initially, a method to capture a simple object will be developed, before attempts to capture more complex (or multiple) objects and real-time object manipulation will be implemented. The project aims to explore to what extent this can be done. \\
The milestones will be: 
\begin{itemize}\itemsep0pt
	\item simple object capture
	\item simple object manipulation
	\item complex object capture
	\item real-time object manipulation.
\end{itemize}
\textbf{The Main Deliverable(s):} \\
Object capture and manipulation software. \\
Dissertation. \\
Poster. \\ \\
\textbf{The Target Audience for the Deliverable(s):} \\
Technical mixed-reality game developers looking for new game mechanics. \\
Enthusiasts interested in new game technologies. \\ \\
\textbf{The Work to be Undertaken:} \\
Research similar attempts at mixed-reality object manipulation. \\
Design a method to capture a cube (or similar object) using Vuforia. \\
Build and apply simple shaders to the object. \\
Expand on above methods to capture more complex objects. \\
Explore and evaluate the extent to which the objects can be manipulated in real-time. \\
Document and report findings. \\ \\
\textbf{Additional Information / Knowledge Required:} \\
Creating and managing Unity projects. \\
How to use Vuforia. \\
Improve understanding of shader usage. \\ \\
\textbf{Information Sources that Provide a Context for the Project:} \\
General Development Page for the HoloLens: \\
\url{https://developer.microsoft.com/en-us/windows/mixed-reality/development} \\
Some related downloadable tools: \\
\url{https://developer.microsoft.com/en-us/windows/mixed-reality/install_the_tools} \\
Vuforia Specific: \\
\url{https://developer.microsoft.com/en-us/windows/mixed-reality/vuforia_development_overview} \\
\url{https://developer.vuforia.com} \\ \\
\textbf{The Importance of the Project:} \\
Mixed-reality is an emerging games technology with great potential for immersive story-telling and innovative game design. This project looks to explore an aspect of that and if successful could be beneficial to those interested in designing such games. \\ \\
\textbf{The Key Challenge(s) to be Overcome:} \\
Lack of similar projects documented. \\
Fairly niche/specialist/obscure software, mostly new territory. \\
\newpage

\section{Report on the Interim Review Meeting}
\ \\ \textbf{SOC10101 Honours Project (40 Credits)} \\                   
\textbf{Week 9 Report} \\ \\
\textbf{Student Name:} Beej Persson \\
\textbf{Supervisor:} Kevin Chalmers \\
\textbf{Second Marker:}  Gregory Leplatre \\
\textbf{Date of Meeting:}  20/11/2017 \\
Can the student provide evidence of attending supervision meetings by means of project diary sheets or other equivalent mechanism?  \textbf{no*} \\
\indent If not, please comment on any reasons presented \\
\textsl{No evidence provided but no indication from supervisor that there was a problem with weekly meeting attendance.} \\ \\ \\
Please comment on the progress made so far \\ \\
\textsl{The work done focused on identifying ways of working with a Hololens, i.e., the output side of the project. This was useful, but the main challenge of the project is object recognition and augmentation, which would have been useful to deal with earlier. This would also have allowed you to engage with relevant literature on the subject.} \\ \\ \\
Is the progress satisfactory? \textbf{unsure} \\
Can the student articulate their aims and objectives? \textbf{Partly} \\
If yes then please comment on them, otherwise write down your suggestions. \\ \\
\textsl{The overall goal (altering the appearance of real-world objects using AR) is clear. What specific alterations, to which objects and in which context remains to be determined. Familiarity with relevant literature will help you specify your project more accurately. Many things are possible, some more complex than others. For this type of project, a good approach would be to consider what operation would have the highest visual impact while having a manageable development cost. There is definitely potential in your project.} \\ \\ \\	
Does the student have a plan of work? \textbf{Yes} \\
If yes then please comment on that plan otherwise write down your suggestions. \\ \\
\textsl{Unfortunately, the plan is limited to work already done.} \\ \\ \\
Does the student know how they are going to evaluate their work? \textbf{partly} \\
If yes then please comment otherwise write down your suggestions. \\ \\
\textsl{The proposed approach is sensible but it is difficult to be more specific at this stage as the functionalities of the system aren’t clearly defined.} \\ \\
Any other recommendations as to the future direction of the project \\ \\
\textsl{With Vuforia, you should be able to capture objects in a scene (in advance) and then track these objects. The acquisition/tracking effectiveness probably depends on the complexity of the object, which you should establish. Once you know the limits of what you will be able to track, you can focus on real-time alterations. Your incremental development approach is suitable, but it needs to be better controlled. Once you know what you are aiming to do, you can define the incremental steps that will lead you there. The core of the research in this area is about achieving realism, which involves lighting and material acquisition. Potentially impactful results can also be achieved by taking slighting different directions. Giving an object a stylised look is one of them. Anecdotally, magic tricks are high impact and can be technically cheap. A magic trick you could implement could be: give a physical object to a user (that you will have previously scanned with Vuforia). Ask them to put it on a table (that you have also previously scanned), and make the object disappear (assuming the user is wearing a Hololens).} \\ \\
\begin{tabbing}
	Signatures: \hspace{1em} \= Supervisor \hspace{2em} \= Kevin Chalmers \\
	\> Second Marker \> Gregory Laplatre \\ \\
	\> Student \>
\end{tabbing}
Please give the student a photocopy of this form immediately after the review meeting; the original should be lodged in the School Office with Leanne Clyde
\newpage


\section{Diary Sheets (or other project management evidence)}
Insert diary sheets here together with any project management plan you have

\section{Appendix 4 and following}
insert content here and for each of the other appendices, the title may be just on a page by itself, the pages of the appendices are not numbered, unless an included document such as a user manual or design document is itself pager numbered.
\end{appendices}

\end{document}
